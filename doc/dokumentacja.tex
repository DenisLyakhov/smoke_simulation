\documentclass[a4paper, 12pt]{scrartcl}
\usepackage[polish]{babel}
\usepackage[utf8]{inputenc}
\usepackage[T1]{fontenc}
\usepackage[top=2.5cm, bottom=2.5cm, left=2.5cm, right=2.5cm]{geometry}
\usepackage{enumerate}
\usepackage{hyperref}


\title{Symulacja rozprzestrzeniania się dymu w sali AGH B1 H.24}
\author{Autorzy: Michał Kowalczyk, Kacper Kontny, Denis Lyakhov}
\date{\today}


\begin{document}
	\maketitle
	
	\begin{section}{Cel}
		Głównym założeniem projektu jest stworzenie trójwymiarowego modelu sali wykładowej, a następnie symulacja rozprzestrzeniania się dymu w wyniku wybuchu pożaru w zadanych warunkach. 
	\end{section}
	
	\begin{section}{Narzędzia}
	Symulacja zostanie przeprowadzona przy użyciu silniku FDS (Fire Dynamics Simulator),
    bazującego się na obliczeniu równań różniczkowych (Navier-Stokes).
    
    \vspace{5mm}
    
    Równanie Navier-Stokes’a - w mechanice płynów - równanie różniczkowe cząstkowe,
    opisujące fizykę cieczy. Mówiąc dokładniej, opisuje zmianę prędkości przepływu w czasie.
    Mając aktualny stan prędkości i zbiór sił, rownania te mogą nam powiedzieć dokładnie,
    jak zmienia się prędkość w każdym nieskończenie małym przedziale czasu.
    Równania Navier-Stokes’a wyglądają następująco:
    
    \begin{equation}
	    \frac{\partial u}{\partial t} = -(u \cdot \nabla)u+\nu\nabla^{2}u+f
	\end{equation}
	
	\begin{equation}
	    \frac{\partial \rho}{\partial t}=-(u\cdot\nabla)\rho+\kappa\nabla^{2}\rho + S
	\end{equation}

	\end{section}

	\begin{section}{Założenia}
		\begin{enumerate}
			\item Model sali wykładowej wzorowany jest na sali H24 znajdującej się w budynku B1 na terenie kampusu Akademii Górniczo-Hutniczej.
			\item Źródłem dymu jest powierzchnia płaska znajdująca się na podłodze pomiędzy stołem wykładowcy a ławkami uczestników, a więc w najniższym punkcie sali.
			\item Dla uproszczenia symulacji intensywność wydobywania się dymu ze źródła nie zmienia się wraz ze spadkiem zawartości tlenu w badanej atmosferze.
			\item Dla danych scenariuszy zostanie wykonany pomiar temperatury w zadanych punktach przestrzeni.
		\end{enumerate}
	\end{section}

	\begin{section}{Przebieg symulacji}
		Wykonane zostaną symulacje różnych scenariuszy:
		\begin{enumerate}
			\item Źródło dymu niezmienne w czasie, zamknięty obieg powietrza w sali
			\item Źródło dymu ugaszone po pewnym czasie, zamknięty obieg powietrza w sali
			\item Źródło dymu ugaszone po pewnym czasie, otworzenie klapy dymowej lub okna sali w pewnej chwili
			\item Źródło dymu ugaszone po pewnym czasie, włączenie wentylatora oddymiającego w pewnej chwili
		\end{enumerate}
		Symulacja zostanie przeprowadzona przy użyciu programu PyroSim bazującego na silniku FDS (Fire Dynamics Simulator)
	\end{section}

	\begin{section}{Wyniki symulacji}
		\textit{Wyniki obejmą:
			\begin{enumerate}
				\item pomiary temperatur oraz czasy tych pomiarów
				\item graficzne porównanie temperatury panującej w pomieszczeniu
				\item graficzne porównanie poziomu zadymienia w funkcji czasu
			\end{enumerate}
		}
	\end{section}
	
	\begin{section}{Opracowanie wyników i wnioski}
		\textit{Opracowanie obejmie:
			\begin{enumerate}
				\item porównanie temperatur na początku, w trakcie i na końcu symulacji
				\item wpływ różnych metod na szybkość oddymiania pomieszczenia
				\item wpływ czasu potrzebnego na wykrycie i ugaszenie źródła dymu na temperatury i zadymienie
			\end{enumerate}
		}
	\end{section}

	\begin{section}{Źródła}
		
		PyroSim Fire Dynamics and Smoke Control by Thunderhead Engineering Consultants, Inc. \textit{\href{https://www.thunderheadeng.com/pyrosim/}{link}}.
		\begin{thebibliography}{15}
			\bibitem{id}
				Eren Algan, \textit{REAL-TIME SMOKE SIMULATION},
				\textit{\href{http://repository.bilkent.edu.tr/bitstream/handle/11693/15609/0006332.pdf?sequence=1&isAllowed=y}{link}}.
				
			\bibitem{id2}
				Jos Stam, \textit{Real-Time Fluid Dynamics for Games}.
				
			\bibitem{id3}
				Marinus Rorbech, \textit{REAL-TIME SIMULATION OF SMOKE USING GRAPHICS HARDWARE},
				\textit{\href{http://image.diku.dk/projects/media/roerbech.04.pdf}{link}}.
		\end{thebibliography}
	\end{section}
	
	

\end{document}